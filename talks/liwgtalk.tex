% on bueler-leopard see
%   /usr/share/doc/latex-beamer/solutions/conference-talks/conference-ornate-20min.en.tex

\documentclass{beamer}

\usetheme{Boadilla}
%\usetheme{Pittsburgh}
\usecolortheme{beaver}

\setbeamercovered{transparent}
\setbeamertemplate{navigation symbols}{} %remove navigation symbols

\usepackage[english]{babel}
\usepackage[latin1]{inputenc}

\usepackage{times}
\usepackage[T1]{fontenc}

\usepackage{animate}

% see http://tex.stackexchange.com/questions/86188/labelling-with-arrows-in-an-automated-way
\usepackage{tikz}
\usepackage{amsmath}

\newif\ifclipme\clipmetrue
\tikzset{labelstyle/.style={LabelStyle/.append style={#1}},linestyle/.style={LineStyle/.append style={#1}}}
\tikzset{LabelStyle/.initial={},LineStyle/.initial={}}

\newcommand{\mathWithDescription}[4][]{{%
    \tikzset{#1}%
    \tikz[baseline]{
        \node[draw=red,rounded corners,anchor=base] (m#4) {$\displaystyle#2$};
        \ifclipme\begin{pgfinterruptboundingbox}\fi
            \node[above of=m#4,font=\strut, LabelStyle] (l#4) {#3};
            \draw[-,red, LineStyle] (l#4) to (m#4);
        \ifclipme\end{pgfinterruptboundingbox}\fi
    }%
}}

\newcommand{\mathWithDescriptionStarred}[3][]{{%
    \clipmefalse%
    \mathWithDescription[#1]{#2}{#3}{\themathLabelNode}%
}}

\newcounter{mathLabelNode}

\newcommand{\mathLabelBox}[3][]{%
   \stepcounter{mathLabelNode}%
   \mathWithDescription[#1]{#2}{#3}{\themathLabelNode}%
   \vphantom{\mathWithDescriptionStarred[#1]{#2}{#3}{\themathLabelNode}}%
}

% math macros
\newcommand\bb{\mathbf{b}}
\newcommand\bbf{\mathbf{f}}
\newcommand\bn{\mathbf{n}}
\newcommand\bq{\mathbf{q}}
\newcommand\bu{\mathbf{u}}
\newcommand\bv{\mathbf{v}}
\newcommand\by{\mathbf{y}}

\newcommand\bQ{\mathbf{Q}}
\newcommand\bV{\mathbf{V}}
\newcommand\bX{\mathbf{X}}

\newcommand\CC{\mathbb{C}}
\newcommand{\DDt}[1]{\ensuremath{\frac{d #1}{d t}}}
\newcommand{\ddt}[1]{\ensuremath{\frac{\partial #1}{\partial t}}}
\newcommand{\ddx}[1]{\ensuremath{\frac{\partial #1}{\partial x}}}
\newcommand{\ddy}[1]{\ensuremath{\frac{\partial #1}{\partial y}}}
\newcommand{\ddxp}[1]{\ensuremath{\frac{\partial #1}{\partial x'}}}
\newcommand{\ddz}[1]{\ensuremath{\frac{\partial #1}{\partial z}}}
\newcommand{\ddxx}[1]{\ensuremath{\frac{\partial^2 #1}{\partial x^2}}}
\newcommand{\ddyy}[1]{\ensuremath{\frac{\partial^2 #1}{\partial y^2}}}
\newcommand{\ddxy}[1]{\ensuremath{\frac{\partial^2 #1}{\partial x \partial y}}}
\newcommand{\ddzz}[1]{\ensuremath{\frac{\partial^2 #1}{\partial z^2}}}
\newcommand{\Div}{\nabla\cdot}
\newcommand\eps{\epsilon}
\newcommand{\grad}{\nabla}
\newcommand{\ihat}{\mathbf{i}}
\newcommand{\ip}[2]{\ensuremath{\left<#1,#2\right>}}
\newcommand{\jhat}{\mathbf{j}}
\newcommand{\khat}{\mathbf{k}}
\newcommand{\nhat}{\mathbf{n}}
\newcommand\lam{\lambda}
\newcommand\lap{\triangle}
\newcommand\Matlab{\textsc{Matlab}\xspace}
\newcommand\RR{\mathbb{R}}
\newcommand\vf{\varphi}


\title[Conservation in free-boundary layers] % (optional, use only with long paper titles)
{Conservation \\ in \\ free-boundary fluid layer \\ models}


\author{Ed Bueler}

\institute[UAF] % (optional, but mostly needed)
{
  Dept of Mathematics and Statistics, and Geophysical Institute\\
  University of Alaska Fairbanks%
}

\date{CESM LIWG 2015}


\begin{document}
\graphicspath{{../images/}{../../talks-public/commonfigs/}}

\begin{frame}
  \titlepage
\end{frame}

\begin{frame}{Outline}
  \tableofcontents
\end{frame}


\section{The problem I'm worried about:}

\subsection{Time-stepping free-boundary fluid layer models.}

\begin{frame}{A fluid layer in a climate}

\begin{center}
\includegraphics[width=\textwidth,keepaspectratio=true]{cartoon-wclimate}
\end{center}

\vspace{-7mm}
  \begin{itemize}
  \item mass conservation for a layer:
      $$h_t + \Div\bq = {\color{blue} f}$$
    \begin{itemize}
    \vspace{-4mm}
    \item[$\circ$] $h$ is a thickness: $h\ge 0$
    \item[$\circ$] mass conservation PDE applies \emph{only where} $h>0$
    \item[$\circ$] $\bq$ is flow (vertically-integrated)
    \item[$\circ$] source ${\color{blue} f}$ is ``climate''; $f>0$ shown downward
    \end{itemize}
  \end{itemize}
\end{frame}


\begin{frame}{Examples}

\includegraphics[width=0.42\textwidth,keepaspectratio=true]{polaris}
\hfill
\includegraphics[width=0.47\textwidth,keepaspectratio=true]{supp4rignot-small}

\small glaciers \hfill ice shelves \& sea ice

\medskip
\includegraphics[width=0.43\textwidth,keepaspectratio=true]{marsh-water}
\hfill
\includegraphics[width=0.44\textwidth,keepaspectratio=true]{tsunami-sendai}

\small tidewater marsh \hfill tsunami inundation

%\vspace{-2mm}
%{\tiny {\color{gray} (Alonso, Santillana \& Dawson, 2008)}}

\medskip
\scriptsize and subglacial hydrology, supraglacial runoff, surface hydrology, \dots
\end{frame}


\begin{frame}{A fluid layer in a climate: \emph{the troubles}}

\begin{center}
\only<1>{\includegraphics[width=\textwidth,keepaspectratio=true]{cartoon-sensitive-three}}
\only<2>{\includegraphics[width=\textwidth,keepaspectratio=true]{cartoon-sensitive-one}}
\only<3>{\includegraphics[width=\textwidth,keepaspectratio=true]{cartoon-sensitive-two}}
\end{center}

\vspace{-18mm}
$$h_t + \Div\bq = f$$

  \begin{itemize}
  \item<1-> $h=0$ \emph{and what else} at free boundary?
     \begin{itemize}
     \item<1->[$\circ$] shape at free boundary depends on both $\bq$ and $f$
     \end{itemize}
  \item<2-> $f<0$ not ``detected'' by model where $h=0$
     \begin{itemize}
     \item<2->[$\circ$] how to do mass conservation accounting?
     \end{itemize}
  \item<3> $f\approx 0$ threshold behavior
     \begin{itemize}
     \item<3>[$\circ$] $h>0$ as soon as $f<0$ switches to $f>0$
     \end{itemize}
  \end{itemize}
\end{frame}


\begin{frame}{Anyone faced these problems before?}

  \begin{itemize}
  \item yes, of course!
    \begin{itemize}
    \item[$\circ$] generic result: \emph{ad hoc} schemes for finite volume/difference mass conservation near the free boundary
    \end{itemize}
  \only<2>{
  \item I don't mind ``\texttt{if \dots then \dots}'' in my code, \emph{but} I want to know what mathematical problem it reflects!
  \item my goals:
    \begin{itemize}
    \item[$\circ$]
       \alert{redefine the problem so free boundary is part of solution}
    \item[$\circ$]
       \alert{use numerical schemes which automate the details}
    \end{itemize}
  }
  \end{itemize}

\only<1>{
\begin{columns}
\begin{column}{0.3\textwidth}
\includegraphics[width=\textwidth,keepaspectratio=true]{JaroschSchoofAnslow2013}

\small glacier ice

on steep terrain

\smallskip
\tiny (Jarosch, Schoof, Anslow, 2013)
\end{column}

\begin{column}{0.6\textwidth}
\includegraphics[width=0.8\textwidth,keepaspectratio=true]{Albrechtetal2011}

\small volume-of-fluid method at ice shelf fronts

\tiny (Albrecht et al, 2011)

\small \medskip
\includegraphics[width=0.6\textwidth,keepaspectratio=true]{jouvet-two-grids}

\small volume-of-fluid method (on fine grid) at glacier surface

\tiny (Jouvet et al 2008)
\end{column}
\end{columns}}

\only<2>{\phantom{foo}

\vspace{29mm}
\phantom{bar}}
\end{frame}


\begin{frame}{Numerical models \emph{must} discretize time}

$$h_t + \Div\bq = f \qquad \to \qquad \frac{H_n - H_{n-1}}{\Delta t} + \Div \bQ_n = F_n$$

  \begin{itemize}
  \item semi-discretize in time: $H_n(x) \approx h(t_n,x)$
  \item the new equation is a ``single time-step problem''
    \begin{itemize}
    \item[$\circ$] a PDE in space \alert{where $H_n>0$}
    \item[$\circ$] this PDE is the ``strong form''
    \end{itemize}
  \item details of flux $\bQ_n$ and source $F_n$ come from time-stepping scheme
    \begin{itemize}
    \item[$\circ$] forward/backward Euler, trapezoid, RK all o.k.
    \item[$\circ$] note low regularity of $h(t,x)$ for $x$ near margin
    \end{itemize}
  \end{itemize}
\end{frame}


\begin{frame}{Weak form incorporates constraint}

  \begin{itemize}
  \item define:
    $$\mathcal{K} = \left\{v \in W^{1,p}(\Omega) \,\Big|\, v\ge 0\right\} = \text{\alert{admissible thicknesses}}$$
  \item define: $H_n \in \mathcal{K}$ solves the \alert{weak single time-step problem} if
    $$\int_\Omega H_n (v - H_n) - \Delta t\, \bQ_n \cdot \grad(v - H_n) \ge \int_\Omega \left(H_{n-1} + \Delta t\, F_n\right) (v - H_n)$$
  for all $v \in \mathcal{K}$
  \small
  \medskip
    \begin{itemize}
    \item[$\circ$] derive this \emph{variational inequality} from:
      \begin{itemize}
      \item[$\diamond$] the strong form \emph{and}
      \item[$\diamond$] integration-by-parts \emph{and}
      \item[$\diamond$] arguments about $H_n=0$ areas
      \end{itemize}
    \end{itemize}
  \end{itemize}
\end{frame}


\begin{frame}{Theorem: weak solves strong}

\emph{Theorem}.  Assume $\bQ_n=0$ when $H_n=0$.  Assume $H_n \in \mathcal{K}$ solves weak single time-step problem and is smooth.  Then
  \begin{enumerate}
  \item ``interior condition'' on set where $H_n>0$:
    $$\frac{H_n - H_{n-1}}{\Delta t} + \Div \bQ_n = F_n$$
  \item on set where $H_n = 0$:
    $$H_{n-1} + \Delta t\, F_n \le 0$$
  \end{enumerate}

\bigskip
re part 2:
  \begin{itemize}
  \item ``climate is negative enough to remove old thickness''
  \item assumption ``$\bQ_n=0$ when $H_n=0$'' is needed
    \begin{itemize}
    \item[$\circ$] \dots yes, we are talking about a \emph{layer}
    \end{itemize}
  \end{itemize}
\end{frame}


\section{Practical consequences:}

\subsection{Limitations in reporting discrete conservation.}

\begin{frame}{Subsets for reporting conservation}
\begin{itemize}
\item suppose $H_n$ solves the weak single time-step problem
\item define
	\begin{align*}
	\Omega_n &= \operatorname{supp} H_n = \left\{H_n(x) > 0\right\} \\
	\Omega_n^r &= \left\{\phantom{\Big|}H_n(x)= 0 \text{ and } H_{n-1}(x) > 0\phantom{\Big|}\right\} \quad \text{\alert{$\leftarrow$ retreat set}}
	\end{align*}
\end{itemize}

\vspace{-2mm}
\begin{center}
\includegraphics[width=1.02\textwidth,keepaspectratio=true]{cartoon-sets}
\end{center}
\end{frame}


\begin{frame}{Reporting discrete conservation}

\begin{itemize}
\item define:
   $$M_n = \int_\Omega H_n(x)\,dx \qquad \text{\emph{mass at time} $t_n$}$$
\item then \vspace{-5mm}
	\begin{align*}
	M_n - M_{n-1} &= \int_{\Omega_n} \mathLabelBox[
    labelstyle={xshift=1cm,yshift=2mm},
    linestyle={out=270,in=90, -latex}
    ]{H_n - H_{n-1}}{$\boxed{\Delta t\, (-\Div\bQ_n + F_n)}$} \,dx + \int_{\Omega_n^r} 0 - H_{n-1} \,dx \\
	   &= \Delta t\, \left(0 + \int_{\Omega_n} F_n \,dx\right) - \int_{\Omega_n^r} H_{n-1} \,dx
	\end{align*}
\item new term:
     $$R_n = \int_{\Omega_n^r} H_{n-1} \,dx \qquad \text{\emph{\alert{retreat loss} during step} $n$}$$
\end{itemize}
\end{frame}


\begin{frame}{Reporting discrete conservation: \emph{claim}}

\begin{itemize}
\item we want to ``balance the books'' for the model user
\item \alert{the retreat loss $R_n$ is not balanced by the climate}
  \begin{itemize}
  \item[$\circ$] yes, $R_n$ is caused by the climate, but we don't know what \emph{computable integral} it balances
  \end{itemize}
\item a numerical model must track \alert{three} time series:
  \begin{itemize}
  \item[$\circ$] mass at time $t_n$: \qquad $M_n = \int_\Omega H_n(x)\,dx$

  \smallskip
  \item[$\circ$] climate over fluid-covered region:
     $$C_n = \Delta t\, \int_{\Omega_n} F_n \,dx \approx \int_{t_{n-1}}^{t_n} \int_{\Omega_n} f(t,x) \,dx\,dt$$
  \item[$\circ$] retreat loss: \qquad $R_n = \int_{\Omega_n^r} H_{n-1} \,dx$
  \end{itemize}
\item now it is balanced:
     $$M_n = M_{n-1} + C_n - R_n$$
\end{itemize}
\end{frame}


\subsection{Numerical weak free boundary solution needed.}

\begin{frame}{I am driven by practical modeling}

\begin{columns}
\begin{column}{0.5\textwidth}
\begin{itemize}
\small
\item practical ice sheet modeling (e.g.~Greenland at right)
\item icy region nearly-fractal and disconnected
\item currently in PISM$^*$:
  \begin{itemize}
  \item[$\circ$] explicit time-stepping
  \item[$\circ$] free boundary by truncation
  \end{itemize}
\item want for PISM:
  \begin{itemize}
  \item[$\circ$] implicit time steps \emph{with free boundary}
  \item[$\circ$] better conservation accounting to user
  \end{itemize}
\end{itemize}

\vspace{10mm}
\quad {\tiny $^*$= Parallel Ice Sheet Model, \texttt{pism-docs.org}}
\end{column}
\begin{column}{0.5\textwidth}
\begin{center}
\only<1>{\includegraphics[width=0.85\textwidth,keepaspectratio=true]{ice-mask-900m}}
\only<2>{\includegraphics[width=0.87\textwidth,keepaspectratio=true]{jako-crop-mask-900m}

\scriptsize PISM whole Greenland at 900m

see Aschwanden C51A-0245 on 12/19}
\end{center}
\end{column}
\end{columns}
\end{frame}


\begin{frame}{Numerical solution of the weak problem}

the weak single time-step problem:
\begin{itemize}
\item is nonlinear because of constraint (even for $\bQ_n$ linear in $H_n$)
\item can be solved by Newton method modified for constraint
  \begin{itemize}
  \item[$\circ$]  reduced set method
  \item[$\circ$]  semismooth method
  \end{itemize}
\item scalable implementations of both in PETSc 3.5
  \begin{itemize}
  \item[$\circ$]  SNESVI class
  \end{itemize}
\end{itemize}
\end{frame}


\begin{frame}{Well-posedness of the weak problem}
\begin{itemize}
\item I've been agnostic on form of $\bQ_n$
  \begin{itemize}
  \item[$\circ$] except ``$\bQ_n=0$ where $H_n=0$'' (i.e.~it's a layer)
  \end{itemize}
\item but form of $\bQ_n$ matters
  \begin{itemize}
  \item[$\circ$] for well-posedness and for numerical solutions
  \end{itemize}
\item cases to study:\small
$$\begin{array}{ll}
  \bQ_n = \bX(x) H_n & \text{\phantom{\Big|}\emph{transported layer}} \\
  \bQ_n = -k \grad H_n & \text{\phantom{\bigg|}\emph{linear diffusion}} \\
  \bQ_n = - \grad(H_n^\gamma) = - \gamma H_n^{\gamma-1} \grad H_n
     & \text{\phantom{\Big|}\emph{porous medium}}\phantom{\bigg|} \\
  \bQ_n = - H_n^\alpha |\grad (H_n+b)|^\beta \grad (H_n+b)\phantom{\bigg|} &
        \hspace{-0.75mm}\begin{array}{ll}
        \text{\emph{shallow ice approx.}} \\
        \text{\& \emph{diffusive shallow water}}
        \end{array} \\
  \bQ_n = \text{\emph{worse (non-local)}}\phantom{\bigg|} & \hspace{-0.75mm}\begin{array}{ll}
        \text{\emph{ice shelf flow} \& \emph{sea ice} \& \dots}
        \end{array}
\end{array}$$

\vspace{-4mm}
\item variational inequality is generally monotone
  \begin{itemize}
  \item[$\circ$] generally coercive if $\bQ_n \sim - \grad H_n$
  \end{itemize}
\end{itemize}
\end{frame}


\begin{frame}{Two numerical examples}

\begin{columns}
\begin{column}{0.4\textwidth}
\only<1>{
\emph{pop quiz}:

\scriptsize
\begin{itemize}
\item same equation

\tiny
\hspace{-10mm} $\frac{H_n-H_{n-1}}{\Delta t} + \Div\bQ_n = f$

\scriptsize
\item same climate $f$
\item same bed shape
\item same

constrained-

Newton

scheme
\end{itemize}

\vspace{5mm}

\normalsize
\alert{how different}

\alert{are the $\bQ_n$?}
}
\only<2>{
\scriptsize
\vspace{20mm}

$\bQ_n = v_0 H_n$

hyperbolic (constant vel.)

\vspace{34mm}

$\bQ_n = - \Gamma |H_n|^{n+2}$

$\phantom{bQ_n = }\, \cdot |\grad h_n|^{n-1} \grad h_n$

highly-nonlinear diffusion
}
\end{column}
\begin{column}{0.6\textwidth}
\hspace{-20mm}
\animategraphics[autoplay,loop,height=3.5cm]{6}{anim/adv/u-}{1}{25}

\medskip
\hspace{-20mm}
\animategraphics[autoplay,loop,height=3.5cm]{6}{anim/sia/u-}{1}{25}
\end{column}
\end{columns}
\end{frame}


\begin{frame}{FIXME: add slides on}

1. truncation in explicit step case is principled

2. where in PISM this layer idea is, and will be, applied?
\end{frame}



\section*{Summary}

\begin{frame}{Terminal slide}

  \begin{itemize}
  \item I'm considering layer flow problems:
    \begin{itemize}
    \item[$\circ$]  model has conservation eqn: $h_t + \Div\bq = f$
    \end{itemize}
  \item suggestions:
    \begin{itemize}
    \item[$\circ$]  \emph{include} constraint on thickness: $h\ge 0$
    \item[$\circ$]  pose single time-step problem \emph{weakly} as variational inequality
    \item[$\circ$]  solve single time-step problem numerically by constrained-Newton method
    \end{itemize}
  \item claim:  for \emph{any} numerical approach,
    \begin{itemize}
    \item[$\circ$]  exact discrete conservation requires tracking \emph{retreat loss}
      \begin{itemize}
      \item[$\diamond$] in addition to computable integrals of climate
      \end{itemize}
    \item[$\circ$]  but it \emph{isn't really possible} except in $\Delta t\to 0$ limit
    \end{itemize}
  \end{itemize}
\end{frame}


\end{document}



\documentclass[final,leqno,onefignum,onetabnum]{siamltex1213bueler}
% siamltex1213bueler.cls is a two or three line change of siamltex1213.cls to permit
% pdflatex to work and not spew warnings

\usepackage{amssymb,amsmath}

\usepackage{times}

% math macros
\newcommand\bv{\mathbf{v}}
\newcommand\bV{\mathbf{V}}
\newcommand\bn{\mathbf{n}}
\newcommand\bq{\mathbf{q}}
\newcommand\bQ{\mathbf{Q}}

\newcommand\CC{\mathbb{C}}
\newcommand{\DDt}[1]{\ensuremath{\frac{d #1}{d t}}}
\newcommand{\ddt}[1]{\ensuremath{\frac{\partial #1}{\partial t}}}
\newcommand{\ddx}[1]{\ensuremath{\frac{\partial #1}{\partial x}}}
\newcommand{\ddy}[1]{\ensuremath{\frac{\partial #1}{\partial y}}}
\newcommand{\ddxp}[1]{\ensuremath{\frac{\partial #1}{\partial x'}}}
\newcommand{\ddz}[1]{\ensuremath{\frac{\partial #1}{\partial z}}}
\newcommand{\ddxx}[1]{\ensuremath{\frac{\partial^2 #1}{\partial x^2}}}
\newcommand{\ddyy}[1]{\ensuremath{\frac{\partial^2 #1}{\partial y^2}}}
\newcommand{\ddxy}[1]{\ensuremath{\frac{\partial^2 #1}{\partial x \partial y}}}
\newcommand{\ddzz}[1]{\ensuremath{\frac{\partial^2 #1}{\partial z^2}}}
\newcommand{\Div}{\nabla\cdot}
\newcommand\eps{\epsilon}
\renewcommand{\grad}{\nabla}
\newcommand{\ihat}{\mathbf{i}}
\newcommand{\ip}[2]{\ensuremath{\left<#1,#2\right>}}
\newcommand{\jhat}{\mathbf{j}}
\newcommand{\khat}{\mathbf{k}}
\newcommand{\nhat}{\mathbf{n}}
\newcommand\lam{\lambda}
\newcommand\lap{\triangle}
\newcommand\Matlab{\textsc{Matlab}\xspace}
\newcommand\RR{\mathbb{R}}
\newcommand\vf{\varphi}

\title{Conservation for fluid layers with free boundaries\thanks{Supported by NASA grant \# NNX13AM16G.}} 

\author{Ed Bueler\thanks{Dept.~of Mathematics and Statistics, and Geophysical Institute, University of Alaska Fairbanks (\texttt{elbueler@alaska.edu}).}}

\begin{document}
\maketitle
\slugger{siap}{xxxx}{xx}{x}{x--x}%slugger should be set to mms, siap, sicomp, sicon, sidma, sima, simax, sinum, siopt, sisc, or sirev

\begin{abstract}
FIXME
\end{abstract}

%\begin{keywords}\end{keywords}

%\begin{AMS}\end{AMS}


\pagestyle{myheadings}
\thispagestyle{plain}
\markboth{ED BUELER}{CONSERVATION FOR FLUID LAYERS WITH FREE-BOUNDARIES}

\section{Introduction}

Problems of this type appear for ice sheets \cite{BLKCB,CDDSV,EgholmNielsen2010,JouvetBueler2012}, shallow water flows over marshes \cite{AlonsoSantillanaDawson}, [Dupuit-Forchheimer APPROX IN GROUNDWATER], subglacial hydrology \cite{AschwandenBuelerKhroulevBlatter,BuelervanPeltDRAFT,Schoofetal2012}, supraglacial runoff (?), sea ice (?), and tsunami run-up (?), among other applications.  Within the context of ice sheet modeling, the complicated business of discrete mass conservation at free boundaries has a small but growing literature \cite{Albrechtetal2011,JaroschSchoofAnslow2013} which suggests the difficulty of the problem, but without a more comprehensive theory, which this paper intends to start.

Let $\Omega \subset \RR^d$ be a bounded open region with sufficiently-regular boundary.  The time-dependent model we consider is usually stated in strong form as follows, including a mass-conservation equation, a fixed-location flux (Neumann) boundary condition, and a constraint:
\begin{align}
u_t &= - \Div \bq + f(u,x,t) &&\text{in } \Omega, \text{ where } u > 0 \label{eq:massconserve} \\
\bq \cdot \bn &= g(x,t) &&\text{on } \partial\Omega, \text{ where } u > 0 \label{eq:fixedneumann} \\
u &\ge 0 &&\text{in } \bar\Omega \label{eq:constraint}
\end{align}
for the \emph{layer thickness} $u(x,t)$ with $x\in \Omega$.  We informally suppose $\bq = \bq(\grad u, u, x, t)$ for now, with more essential detail below.

In the time-dependent support of $u$ we say that the layer is present, and that it is absent in the complement.  Finding the evolving ``free'' boundary between the support and its complement is a critical aspect of problems of this type.

Constraint \eqref{eq:constraint} comes from the meaning of $u$ as a layer \emph{thickness}, which must be nonnegative.  Conservation equation \eqref{eq:massconserve}, which is sometimes called a ``St.~Venant'' equation [BECAUSE OF FIXME], is only intended to apply, in this strong form, where the layer exists ($u>0$).  The same applies to the boundary condition \eqref{eq:fixedneumann}, as it can only make sense in locations where $u>0$, at least in the generic case $g\ne 0$.  We will see that well-posedness of the problem \eqref{eq:massconserve}--\eqref{eq:constraint} in fact requires a weak formulation, in which these caveats are replaced by a precise specification of admissible functions.

We actually work with the (time) semi-discretized problem, which will be a stated as a weak problem in $W^{1,p}(\Omega)$, with attention to well-posedness, below.  For now, let $\{t_n\}$ be a sequence of increasing times.  The semi-discretized problem is for the new values $u_n(x) \approx u(x,t_n)$, given the old values $u_{n-1}(x)$.  Here is the strong form, corresponding to \eqref{eq:massconserve}--\eqref{eq:constraint} above:
\begin{align}
\frac{u_n - u_{n-1}}{\Delta t} &= - \Div \bQ_n + F_n &&\text{in } \Omega, \text{ where } u_n > 0 \label{eq:semimassconserve} \\
\bQ_n \cdot \bn &= G_n &&\text{on } \partial\Omega, \text{ where } u_n > 0 \label{eq:semifixedneumann} \\
u_n &\ge 0 &&\text{in } \bar\Omega \label{eq:semiconstraint}
\end{align}

Here $F_n(u_n,x)$, $\bQ_n(\grad u_n,u_n,x)$, $G_n(x)$ are rather general functions coming from the semi-discretization procedure, and including various functions of the old values.  For example, in the simplest implicit case of a backward Euler scheme applied to \eqref{eq:massconserve}--\eqref{eq:constraint}, we would have $F_n = f(u_n,x,t_n)$ and $\bQ_n = \bq(\grad u_n,u,x,t_n)$.  In the case of a trapezoid rule, however,
\begin{align*}
F_n &= \frac{1}{2} f(u_n,x,t_n) + \frac{1}{2} f(u_{n-1},x,t_{n-1}) - \frac{1}{2} \Div \bq(\grad u_{n-1},u_{n-1},x,t_{n-1}),
\end{align*}
and $\bQ_n = \frac{1}{2} \bq(\grad u_n,u_n,x,t_n)$.  Thus $F$ generally ``absorbs'' various terms evaluated at time $t_{n-1}$ and time $t_n$ also.

FIXME: we can extend $G_n$ by zero to the whole of $\partial \Omega$ and not change anything.  On the flip side, as a modeler you are free to put $G_n$ zero anywhere on the boundary, but if it is nonzero then it must be that the solution $u_n$ is actually positive on that part of the boundary

Decompose $\Omega$ into three disjoint regions based on $u_n$ and $u_{n-1}$:
\begin{align*}
\Omega_n &= \left\{x \in \Omega \,\big|\, u_n(x)>0\right\}, \\
\Omega_n^r &= \left\{x \in \Omega \,\big|\, u_n(x)=0 \text{ and } u_{n-1}(x) > 0\right\}, \\
\Omega_n^0 &= \left\{x \in \Omega \,\big|\, u_n(x)=0 \text{ and } u_{n-1}(x) = 0\right\},
\end{align*}
so that $\Omega = \Omega_n \cup \Omega_n^r \cup \Omega_n^0$.  Here the superscript ``$r$'' stands for ``retreat''.\footnote{At this point a symmetry has been broken.  We could have decomposed $\Omega= \Omega_n \cup \Omega_n^a \cup \Omega_n^0$ where $\Omega_n^a = \{u_n(x) > 0 \text{ and } u_{n-1}(x) = 0\}$ is the ``advance'' set.  As far as we can tell the resulting alternate theory offers no advantages \dots}  See Figure \ref{fig:domains}.

The boundary of the support $\Omega_n$ of $u_n$ decomposes into the part where a fixed (Neumann) condition applies, and a part which is the free boundary,
\begin{align*}
\partial\Omega_n &= \Gamma_n^N \cup \Gamma_n^0
\end{align*}
(superscript ``$N$'' stands for ``Neumann'').  Specifically, $\Gamma_n^0 = \Omega \cap \partial \Omega_n$ and $\Gamma_n^N = \partial \Omega \cap \partial \Omega_n$, and along $\Gamma_n^N$ the flux condition \eqref{eq:semifixedneumann} applies.  We will show [WILL WE?  NO, WE CAN'T!] that along the free boundary $\Gamma_n^0$ we have both $u_n=0$ and $\bQ_n = 0$.

\begin{figure}[ht]
\vspace{1.0in}
\centerline{FIXME}
\vspace{1.0in}
%\includegraphics[width=5.0in,keepaspectratio=true]{figs/cheb2dgrid}
\caption{At times $t_{n-1},t_n$ the layer has thicknesses $u_{n-1},u_n$, and this implies a decomposition of $\Omega$.}
\label{fig:domains}
\end{figure}

Define
\begin{equation}
M_n = \int_\Omega u_n(x)\,dx,
\end{equation}
which we call the \emph{(total) mass at time} $t_n$, and define
\begin{equation}
R_n = \int_{\Omega_n^r} u_{n-1}\,dx,
\end{equation}
which we call the \emph{retreat loss at time} $t_n$.

The mass and the retreat loss at time $t_n$ are related, of course.  By \eqref{eq:semimassconserve} we have
\begin{align}
M_n - M_{n-1} &=  - \int_{\Omega_n^r} u_{n-1}\,dx + \int_{\Omega_n} (u_n - u_{n-1})\,dx \label{eq:massstep} \\
   &= - \int_{\Omega_n^r} u_{n-1}\,dx + \Delta t \int_{\Omega_n} (- \Div \bQ_n + F_n) \,dx \notag \\
   &= - R_n + \Delta t \int_{\Gamma_n^N} G_n + \Delta t \int_{\Omega_n} F_n\,dx \notag
\end{align}
because $\bQ_n=0$ along $\Gamma_n^0$.

Conceptually, at points $x$ within $\Omega_n^r$ the time $\bar t(x)$ at which the unknown continuous-time solution $u(x,t)$ first becomes zero varies over $\Omega_n^r$.  While such ``details'' can disappear in the limit $\Delta t \to 0$, practical (i.e.~numerical) models necessarily have discrete time, while they also (e.g.~in the context of climatic modeling) must necessarily conserve discrete mass.  Our practical major point is this restatement of equation \eqref{eq:massstep} as an operational statement about discrete-time models:
\begin{quote}
\emph{one must keep time series for $R_n$, in addition to $\int_{\Gamma_n^N} G_n$ and $\int_{\Omega_n} F_n$, in order to provide auditable mass conservation.}
\end{quote}
More informally,
\begin{quote}
\emph{the retreat loss is not balanced by the source term or a boundary integral,}
\end{quote}
admitting that the retreat loss $R_n$ vanishes entirely in the $\Delta t\to 0$ limit.

The retreat area $|\Omega_n^r|$ can be of essentially arbitrary size.  For example, in a varying climate a large area of thin ice sheet or sea ice can melt, or a large area of a layer of surface water on ground can evaporate, in the short duration of one time step.  In these example cases the mass of water in all phases is conserved, but climatic models additionally attempt to conserve masses of the phases of water separately, as these phases have different physical properties relevant to earth system dynamics (e.g.~snow and ice have higher albedo than liquid ocean).  In multiphysics models with at least one fluid having a moving free boundary, auditable mass conservation requires a broader conception of discrete-time conservation than is currently understood.


\section{Regarding the flux}

FIXME:  want property that for $X$ an open set,
\begin{equation}
u_n=0 \text{ on } X \quad \implies \quad \bQ_n=0 \text{ on } X  \label{eq:vanishingQn}
\end{equation}

FIXME:  want maximum principle property that
\begin{equation}
v + \alpha\, (\Div \bQ_n)(\grad v,v,x) > 0 \text{ on } X \quad \implies \quad v > 0 \text{ on } X  \label{eq:maxprincQn}
\end{equation}
for all $\alpha>0$ because the strong form says $u_n + \Delta t\, \Div \bQ_n = (u_{n-1} + \Delta t\, F_n)$ and we want to conclude the contrapositive of \eqref{eq:maxprincQn} in $\Omega_n^0$ and $\Omega_n^r$:
\begin{equation}
F_n \le 0  \text{ on } \Omega_n^0  \label{eq:inequalityonzero}
\end{equation}
\begin{equation}
u_{n-1} + \Delta t\, F_n \le 0  \text{ on } \Omega_n^r  \label{eq:inequalityonretreat}
\end{equation}

FIXME: if $v \in W^{1,p}(\Omega)$ then, for $1/p + 1/q = 1$,
    $$\bQ_n(\grad v,v,x) \in L^q(\Omega)$$ 

\section{Weak formulation of a time-step}  The strong form \eqref{eq:semimassconserve}--\eqref{eq:semiconstraint} is generally understood to be inadequate as a description of the solutions $u_n(x)$, because the free boundary is not organically included in the problem statement, and also because the space of admissible solutions is not specified.  Here we both specify the appropriate function spaces and propose a weak form, a variational inequality \cite{Friedman,KinderlehrerStampacchia} for \eqref{eq:semimassconserve}--\eqref{eq:semiconstraint}.  This weak form can be proven to be well-posed in some cases.  In more cases the weak form can be shown to imply the strong form where the layer exists (interior condition).

We start by arguing informally for why the weak form, a non-obvious variational inequality, should hold.\footnote{Though our idea is that the weak form is more fundamental, nonetheless mathematical history, if not human frailty, has made the strong form much more prominent in applications and in the climate modeling literature relevant to our problem.}  Our argument uses facts which are at least intuitively true in the subsets of the decomposition $\Omega = \Omega_n \cup \Omega_n^r \cup \Omega_n^0$.

Suppose $v\ge 0$ is sufficiently smooth on $\Omega$.  Using the decomposition and the divergence theorem (Green's theorem),
\begin{align*}
-\int_{\Omega} \bQ_n \cdot \grad(v-u_n) &= -\int_{\Omega_n} \bQ_n \cdot \grad(v-u_n) - \int_{\Omega_n^r \cup \Omega_n^0} \bQ_n \cdot \grad(v-u_n) \\
  &= \int_{\Omega_n} (\Div \bQ_n) (v-u_n) - \int_{\Omega_n} \Div \left(\bQ_n (v-u_n)\right) \\
  &\qquad\quad + \int_{\Omega_n^r \cup \Omega_n^0} (\Div \bQ_n) (v-u_n) - \int_{\Omega_n^r \cup \Omega_n^0} \Div \left(\bQ_n (v-u_n)\right) \\
  &= \int_{\Omega_n} (\Div \bQ_n) (v-u_n) - \int_{\Gamma_n^N} G_n (v-u_n) \\
  &\qquad\quad + \int_{\Omega_n^r \cup \Omega_n^0} (\Div \bQ_n) (v-u_n)
\end{align*}
because
       $$\int_{\Gamma_n^0} (\bQ_n \cdot \bn) (v-u_n) = 0,$$
and using \eqref{eq:semifixedneumann}.  By \eqref{eq:semimassconserve} where the layer exists, namely in $\Omega_n$, we have
\begin{align}
-\int_{\Omega} \bQ_n \cdot \grad(v-u_n) &= \int_{\Omega_n} \left(F_n - \frac{u_n - u_{n-1}}{\Delta t}\right) (v-u_n) - \int_{\partial \Omega} G_n (v-u_n) \label{eq:equalitybeforeVI} \\
  &\qquad\quad + \int_{\Omega_n^r \cup \Omega_n^0} (\Div \bQ_n) (v-u_n). \notag
\end{align}
Here we have used the fact that $G_n$ extends by zero to the whole of $\partial \Omega$.

However, by \eqref{eq:vanishingQn} on $\Omega_n^0$,
    $$\int_{\Omega_n^0} (\Div \bQ_n) (v-u_n) = \int_{\Omega_n^0} (0) (v-u_n) \ge \int_{\Omega_n^0} \left(F_n - \frac{u_n - u_{n-1}}{\Delta t}\right) (v-u_n)$$
because in fact $u_n=u_{n-1}=0$ and $v-u_n = v \ge 0$ on $\Omega_n^0$, and because \eqref{eq:inequalityonzero} says that $F_n \le 0$ on $\Omega_n^0$.  Almost the same, by \eqref{eq:vanishingQn} on $\Omega_n^r$,
    $$\int_{\Omega_n^r} (\Div \bQ_n) (v-u_n) = \int_{\Omega_n^r} (0) (v-u_n) \ge \int_{\Omega_n^r} \left(F_n - \frac{u_n - u_{n-1}}{\Delta t}\right) (v-u_n)$$
because in fact $u_n=0$ and $v-u_n = v \ge 0$ on $\Omega_n^0$, and because \eqref{eq:inequalityonretreat} says that $F_n - (u_n - u_{n-1})/\Delta t = F_n + u_{n-1}/\Delta t \le 0$ on $\Omega_n^0$.  Thus if we return to \eqref{eq:equalitybeforeVI} we have
\begin{align}
-\int_{\Omega} \bQ_n \cdot \grad(v-u_n) &\ge \int_{\Omega_n} \left(F_n - \frac{u_n - u_{n-1}}{\Delta t}\right) (v-u_n) - \int_{\partial \Omega} G_n (v-u_n) \label{eq:essentiallyVI} \\
  &\qquad\quad + \int_{\Omega_n^r \cup \Omega_n^0} \left(F_n - \frac{u_n - u_{n-1}}{\Delta t}\right) (v-u_n). \notag
\end{align}

We want to be able to pose our problem (weakly) so that the decomposition $\Omega = \Omega_n \cup \Omega_n^r \cup \Omega_n^0$ is not needed to pose the problem.  But \eqref{eq:essentiallyVI} can be written without that decomposition!:
\begin{equation}
-\int_{\Omega} \bQ_n \cdot \grad(v-u_n) \ge \int_{\Omega} \left(F_n - \frac{u_n - u_{n-1}}{\Delta t}\right) (v-u_n) - \int_{\partial \Omega} G_n (v-u_n) \label{eq:morallytheVI}
\end{equation}
This is our variational inequality weak form, in which the free boundary does not appear in posing the problem.

\begin{definition}  Fix $p>1$.  Let
    $$\mathcal{K} = \left\{v \in W^{1,p}(\Omega) \,\big|\, v(x) \ge 0 \text{ for all } x \in \Omega\right\}.$$
\end{definition}

\begin{definition}  We say $u_n \in \mathcal{K}$ \emph{solves the weak time-step problem} if 
\begin{align}
\int_{\Omega} u_n (v-u_n) - \Delta t\, \bQ_n \cdot \grad(v-u_n) + &\Delta t \int_{\partial \Omega} G_n (v-u_n) \ge \label{eq:theVI} \\
  &\qquad\qquad \int_{\Omega} \left(\Delta t F_n + u_{n-1}\right) (v-u_n). \notag
\end{align}
\end{definition}




\section{Conclusion}  FIXME


%         References
\bibliography{ice-bib}
\bibliographystyle{siam}

%\Appendix
%\section{FIXME}


\end{document}
